\BiAppChapter{正态分布}{}

要说到概率论与统计里面最有趣的公式, 恐怕正态分布当之无愧可以当选. 这个分布的 $p.d.f$ 虽然形式咋看之下不是非常优美, 但是, 这个函数却在隐隐之中可以让人看见自然界的秩序所在. 我们在课堂上学习这个分布的时候, 却并没有计较这个函数的来历, 我们想先介绍一下和这个函数的一些故事:

	\BiSection{正态分布的历史}{}
		正态分布的概率密度函数为:

		\begin{equation}
			f(x) = \frac{1}{\sqrt{2 \pi} \sigma} e^{-\frac{(x-\mu)^2}{2 \sigma^2}} 	
		\end{equation}
	
		
		很难想象这个复杂的函数是如何被发现的. 
		
		历史上, 这个函数的第一次发现来源于棣莫弗(De Moivre)对二项分布的近似计算, 即, 对于随机变量$X \sim B(n, p)$ 求出当 $n$ 比较大的时候 $X$ 所服从的分布. 这个问题的结论我们也比较熟悉, 当$n\rightarrow\infty$ 的时候, $X$ 服从的是正态分布, 其中正态分布的期望值为 $\mu = np $, 方差为 $\sigma^2 = np(1-p)$. 然而, 我们现在并不只道为什么. 

		\BiSubsection{对于二项分布的近似计算}{}
				上面这个问题的有一个用二项分布表示的表达式,对于其中的一个特例$p = \frac{1}{2}$
				
					\begin{equation}\label{ch1:binom}
						P(X = k) = \binom{k}{n} (\frac{1}{2})^n
					\end{equation}
				我们可以换一个形式来写这个式子				
				
					\begin{equation}\label{ch1:binom}
						P(X = \frac{n}{2} \pm d) = \binom{\frac{n}{2} \pm d}{n} (\frac{1}{2})^n
					\end{equation}		
								
				其中$\binom{k}{n} = \frac{n!}{k! (n-k)!}$ 表示二项式系数. 棣莫弗在计算这个表达式的时候利用了一个近似的公式, 即斯特林(Stirling)公式:
				
					\begin{equation}
						n! \sim \sqrt{2\pi n} (\frac{n}{e})^n
					\end{equation}
				
								
				用上面的这个式子代入 \ref{ch1:binom} 得到
					\begin{equation}
						P(X = \frac{n}{2}) = \frac{n!}{(\frac{n}{2}!)^2} (\frac{1}{2}) ^n \approx \sqrt{\frac{2}{\pi n}}
					\end{equation}
					
				然后, 不难得到, 
					\begin{equation}
						P(X = \frac{n}{2} + d) = \frac{2}{\sqrt{2 \pi n}} e^{-\frac{2d^2}{n}}
					\end{equation}
				这个时候, 其实已经可以发现, 这个式子已经和正态分布的概率密度函数非常接近了. 这个就是正态分布最早的发现, 在这个发现的基础上, 拉普拉斯把这个结论推广到了 $p \neq \frac{1}{2}$ 的情况. 
				
				于是, 就有了棣莫弗-拉普拉斯(De Moivre-Laplace)中心极限定理:
				
				\begin{theorem}
					\begin{equation}						
						\lim_{n \rightarrow \infty} P\left\{\frac	{X_n -np}{\sqrt{np(1-p)}} \le x\right\} = \int_{-\infty}^{x} \frac{1}{\sqrt{2 \pi}} e^{-\frac{x^2}{2}} dt
					\end{equation}
				\end{theorem}
				
				
				我们接下来可以证明一个更加一般的结论.
				
	\BiSection{中心极限定理}{}
		我们课堂上所介绍的中心极限定理指的是林登伯格-莱维(Lindenberg-Levy)中心极限定理. 
		\begin{theorem}
			如果$n$个随机变量$X_i$有有限的期望$\mu$与方差$\sigma^2$, 且同分布, 那么
			\begin{equation}
				\frac{\sum_{i=1}^n X_i - n\mu}{\sqrt{n} \sigma} \sim N(0,1)
			\end{equation}
		\end{theorem}
		中心极限定理的证明需要一些工具, 我们下面先给出他们.
		
		\begin{definition}			
			设一个随机变量$ X$的$p.d.f$ 为$f(X)$, 那么定义这个随机变量的特征函数为:
			\begin{equation}
				\varphi(\omega) = \int_{-\infty}^{+\infty} f(t) e^{i\omega t} dt
			\end{equation}
		\end{definition}
		
		由复变函数的相关知识, 可以知道, 特征函数其实就是概率密度函数的傅立叶变换的共轭复数. 以及特征函数和概率密度函数之间存在双射关系.
		那么, 我们在计算两个独立的随机变量的和的时候, 就只需要计算这两个函数的特征函数的卷积, 然后再做傅里叶逆变换就可以求得加和后的变量所满足的关系.
		
		\begin{proof}
				对于任意的一组满足中心极限定理前提的随机变量$X_i$,设$X_i-\mu$的特征函数为$\varphi(t)$, 有如下的式子成立:
		
				\begin{equation}
					\varphi(0)' = i E(X_i - \mu) = 0
				\end{equation}
		
				\begin{equation}
					\varphi(0)'' = i^2 E((X_i-\mu)^2) = -[D(X_i-\mu) - [E(X_i - \mu)]^2] = -\sigma^2
				\end{equation}
		
				所以, $X$ 的特征函数可以在 $\omega = 0$ 附近泰勒展开为 $\varphi(\omega) = 1 - \frac{1}{2} \sigma^2 \omega^2 + o(n^2)$
		
				那么, 对于 $n$ 个独立同分布的随机变量求和, 和的特征函数就为他们所服从的随机变量特征函数的$n$次幂, 即 
				
				\begin{equation}
					\begin{aligned}[]						
						\left[\varphi(\frac{\omega}{\sigma \sqrt{n}})\right]^n & =  \left[1 - \frac{1}{2n} \omega^2 + o(\omega^2)\right]^n \\
						& =  e^{-\frac{\omega^2}{2}}
					\end{aligned}				
				\end{equation}
				
				而标准正态分布的概率密度函数$\Phi(x)$ 对应的特征函数就是$e^{-\frac{\omega^2}{2}}$.
				所以, 命题得证.							
		\end{proof}
		
		正态分布是如此的优美,大部分的随机变量, 居然都可以在中心极限定理的作用下统一为一个随机变量.
					